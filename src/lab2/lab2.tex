\documentclass[a4paper]{article}

\usepackage{polski}
\usepackage[utf8]{inputenc}

\usepackage[export]{adjustbox}
\usepackage{scrextend}
\usepackage{amsfonts}
\usepackage{amsmath}
\usepackage{svg}

\usepackage{geometry}
\geometry{a4paper, left=15mm, top=30mm, right=15mm, bottom=20mm}

\usepackage{gensymb}
\usepackage{graphicx} 
\usepackage{isotope}
\usepackage{array}
\usepackage{float}
\usepackage{titlesec}
\usepackage{fancyhdr}
\usepackage{multirow}

\usepackage{hyperref}
\usepackage{sectsty}
\usepackage{enumitem}
\usepackage{listings}
\usepackage[labelformat=simple]{subcaption}
\usepackage{xcolor,colortbl}
\usepackage{animate}

\usepackage{karnaugh-map}

\sectionfont{\normalfont\huge\sectionrule{0pt}{0pt}{-6pt}{1pt}}
\subsectionfont{\normalfont\LARGE}
\subsubsectionfont{\normalfont\Large}

\pagestyle{fancy}
\fancyhf{}
\fancyhead[LE,LO]{\Large Łukasz Kwinta, Kacper Kozubowski, Ida Ciepiela}
\fancyhead[LE,RO]{\Large Układ odliczający}
\fancyfoot[CE,CO]{\Large\thepage}

\renewcommand{\footrulewidth}{1pt}
\renewcommand{\headrulewidth}{1pt}

\definecolor{Gray}{gray}{0.85}
\definecolor{LightGray}{gray}{0.95}

\newcolumntype{a}{>{\columncolor{Gray}}c}
\newcolumntype{b}{>{\columncolor{white}}c}

\hypersetup{
    colorlinks,
    citecolor=black,
    filecolor=black,
    linkcolor=black,
    urlcolor=black
}

\counterwithin{table}{section}
\counterwithin{figure}{section}

\title{\fontsize{30pt}{30pt}\selectfont Laboratorium 2 \\ Układ odliczający}
\author{\fontsize{20pt}{20pt}\selectfont Łukasz Kwinta, Kacper Kozubowski, Ida Ciepiela}
\date{marzec 2024}

\begin{document}
\maketitle
\pagebreak
\large
\tableofcontents

\pagebreak
\section{Cel zadania}
\Large
Korzystając wyłącznie z wybranych przerzutników oraz dowolnych bramek logicznych, proszę zaprojektować czterobitowy układ TIMER, odmierzający ustawiany za pomocą przełączników czas (od 0 do 15).
\\\\
Po wciśnięciu przycisku STRAT, układ rozpoczyna odmierzanie czasu do tyłu (proszę dobrać częstotliwość tak, aby efekt był dobrze widoczny na ekranie). 
\\\\
Po wyzerowaniu się licznika czasu, układ powinien się zatrzymać i włączyć alarm świetlny wykorzystujący diodę LED. Po ponownym wciśnięciu przycisku START, 
układ powinien wyłączyć alarm i ponownie rozpocząć odmierzanie ustawionego na przełącznikach czasu.
Aktualny wskazywany przez układ czas proszę pokazywać na wyświetlaczach siedmiosegmentowych.

\section{Idea rozwiązania}
\Large 


\end{document}


